\documentclass[a4paper,11pt]{article}
\usepackage{amsmath}
%\usepackage{catoptions}
\usepackage{hyperref}
\usepackage{graphicx}
\usepackage{listings}
\usepackage{color}

\definecolor{gray}{rgb}{0.5,0.5,0.5}
\newcommand{\convolution}{\ensuremath{+\negmedspace\negmedspace\negmedspace\times}}
\newcommand{\highlightColor}{red}

\makeatletter
%% Provide \Autoref; the \autoref with a printed capital
\def\figureautorefname{figure}
\def\tableautorefname{table}
\def\partautorefname{part}
\def\appendixautorefname{appendix}
\def\equationautorefname{equation}
\def\AMSautorefname{equation}
\def\theoremautorefname{theorem}
\def\enumerationautorefname{case}
\def\Autoref#1{%
  \begingroup
  \edef\reserved@a{\cpttrimspaces{#1}}%
  \ifcsndefTF{r@#1}{%
    \xaftercsname{\expandafter\testreftype\@fourthoffive}
      {r@\reserved@a}.\\{#1}%
  }{%
    \ref{#1}%
  }%
  \endgroup
}
\def\testreftype#1.#2\\#3{%
  \ifcsndefTF{#1autorefname}{%
    \def\reserved@a##1##2\@nil{%
      \uppercase{\def\ref@name{##1}}%
      \csn@edef{#1autorefname}{\ref@name##2}%
      \autoref{#3}%
    }%
    \reserved@a#1\@nil
  }{%
    \autoref{#3}%
  }%
}


\makeatother

 
\author{Maarten Inja (5872464) \and Patrick de Kok (5640318)}
\title{MLPR: Lab 2: Naive Bayes SPAM detector}


\lstset{
  language=Octave,
  basicstyle=\footnotesize,
  numbers=left,
  numberstyle=\tiny\color{gray},
  stepnumber=5,
  showspaces=false,
  showstringspaces=true
  showtabs=false,
  frame=single,
  title=\lstname,
  breaklines=true,
}
\begin{document}
\maketitle

In this report, we describe how we have build a naive Bayes classifier, which will classify text messages as either ``spam'' (unwanted messages), or ham (wanted messages).  

For the implementation we have used Python (version 2.7) with standard libraries.  We do not have a copy of Matlab on our PCs, and thus wanted to run the provided code on Octave, an free and open implementation of Matlab.  The provided code proved not to be compatible with the version of Octave included in our Linux repositories.  Although we have upgraded our Octave version from 3.2 to 3.6, as well as the I/O libraries, the code did not want to run.  After talking with Gwenn Englebienne, we have decided to implement the algorithm in Python. 

\section*{Feature Selection}
The feature vector $\vec{x}$ consists of those words which 
help to distinguish between spam and ham emails the most. 

In order to find these words we wrote some code in \textit{features.py} 
to compute the probability of a word $w$ given a class. Using Bayes' rule
we can determine this quite easy: $P(w|C_k) = \frac{P(C_k|w)P(w)}{P(C_k)}$.

The prior $P(C_l)$ is based on the statistics the MAAWG collected over the 
third quarter of 2011.
\url{http://www.maawg.org/sites/maawg/files/news/MAAWG_2011_Q1Q2Q3_Metrics_Report_15.pdf}. This results in $P(SPAM) = 0.88$, and $P(HAM) = 0.12$.


\section*{Naive Bayes classifier}


\end{document}
